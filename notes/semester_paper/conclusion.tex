\chapter{Conclusion}

In this semester papers, part of theory of data completion is reviewed. The
main work is devoted to build a framework to test several R packages for
imputation methods on the FLAS data set. With this data set, it appears that
K-nearest neighbor imputation works fairly well, although its implementation
might throw frustrating low level errors (segmentation faults). Except
for \texttt{softImpute}, all methods have the same order of error (distance
between the imputed and true value). In practice, they could unfortunately not
be use as black-box as most data matrix have colinear dimensions which
constitutes an issue for all the algorithms based on regressions.

Finally, as departing words, one should not forget why these technique
exists. From \cite{schafer2002missing},

\begin{quote}
  With or without missing data, the goal of a statistical procedure should be
  to make valid and efficient inferences about a population of interest -- not to
  estimate, predict, or recover missing observations nor to obtain the same
  results that we would have seen with complete data.
\end{quote}


%%% Local Variables: ***
%%% mode:latex ***
%%% TeX-master: "semester_paper_sfs.tex"  ***
%%% End: ***
%%% reftex-default-bibliography: ("biblio.bib")
